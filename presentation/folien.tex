\documentclass{beamer}
\usepackage{beamerthemeshadow}
\usepackage[ngerman]{babel}
\usepackage[utf8]{inputenc}
%Tabellen:
	\usepackage{slashbox} %schrägstich in tabelle möglich
	\usepackage{multirow}
%Mathematik:
	\usepackage{dsfont} %Symbole
	\usepackage{amsmath} %Umgebung
	\usepackage{amssymb} %Symbole
	\usepackage{bbm} %doppelstreifen bei buchstaben (zb symbol für ganze zahlen \mathbbm{Z})
%Grafiken:
	\usepackage{graphics}
	\usepackage{graphicx}
	\usepackage{picinpar} %bilder so einfügen, dass text um bilder weiterläuft
	\usepackage{float} %\begin{figure}[H] => Grafik wird HIER eingefügt!
%Pseudo Code:
	\usepackage{algorithmic}
	\usepackage{algorithm}
	
\DeclareMathOperator*{\argmin}{arg\,min}

\beamersetuncovermixins{\opaqueness<1>{25}}{\opaqueness<2->{15}}
\setbeamertemplate{footline} %seitenzahl rechts unten
{%
  \leavevmode%
 \begin{beamercolorbox}%
    [wd=.5\paperwidth,ht=2.5ex,dp=1.125ex,leftskip=.3cm,rightskip=.3cm]%
    {author in head/foot}%
    \usebeamerfont{author in head/foot}%
    \hfill\insertshortauthor
  \end{beamercolorbox}%
  \begin{beamercolorbox}%
    [wd=.5\paperwidth,ht=2.5ex,dp=1.125ex,leftskip=.3cm ,rightskip=.3cm]%
    {title in head/foot}%
    \usebeamerfont{title in head/foot}%
    \insertshorttitle\hfill\insertframenumber{}/\inserttotalframenumber
  \end{beamercolorbox}%
}%

\beamertemplatenavigationsymbolsempty % Abschalten der kleinen Navigationsleiste am unteren Rand

\begin{document}
\title{Online-Marketing der Interhyp AG}
\subtitle{Analyse von Tracking-Daten} 
\author[D. Fuckner \& M. Vogler]{Daniel Fuckner\\Markus Vogler\\Betreuer: Fabian Scheipl}
\institute{Statistisches Consulting\\Institut für Statistik\\Ludwig-Maximilians-Universität München}
\date{12.08.2014} 

\begin{frame}
	\titlepage
\end{frame}

\begin{frame}\frametitle{Inhaltsverzeichnis}
	\tableofcontents[hideallsubsections]
\end{frame}

\section{Einleitung} 

\begin{frame}\frametitle{Einleitung} 
	\begin{itemize}
		\item Interhyp AG ist Vermittler für private Baufinanzierungen
		\item Primäres Ziel des Marketing ist die Kundenakquise
		\item Etwa $80 \%$ aller Kundenanträge werden online abgeschickt
		\item Online-Marketing verfügt über verschiedene Kanäle
		\item Refined Labs GmbH ist verantwortlich für das Online-Tracking der Werbekampagnen der Interhyp AG
	\end{itemize}
\end{frame}

\begin{frame}\frametitle{Entstehung eines Funnels (Quelle: Interhyp AG)}
	\includegraphics[scale=0.38]{customerJourney.png}\\
	\centering Unterschiede zwischen konvertierten und nicht-konvertierten Funnels?
\end{frame}
\section{Deskriptive Analyse}

\begin{frame}\frametitle{Inhalt}
	\tableofcontents[currentsection,hideallsubsections]
\end{frame}

\begin{frame}\frametitle{Beispiel für einen Auszug aus der Datenbank}
	\begin{table}[H]
		\begin{center}
			\begin{tabular}{|c|l|c|c|c|c|}
				\hline
				ID & Campaign 									 & Transaction & Position & ... \\ \hline\hline
				1  & Affiliate - Partnerprogramm & 0					 & 1		    & ... \\ \hline
				1  & SEM - Brand                 & 0					 & 2		    & ... \\ \hline
				1  & Direct                      & 0					 & 3		    & ... \\ \hline
				1  & Direct                      & 1					 & 4		    & ... \\ \hline
				2  & Display                     & 0					 & 1		    & ... \\ \hline
				2  & SEM - Generisch             & 0					 & 2		    & ... \\ \hline
				2  & Social Media                & 0					 & 3		    & ... \\ \hline
			\end{tabular} 
		\end{center}
	\end{table}
\end{frame}

\begin{frame}\frametitle{Datenlage} 
	\begin{itemize}
		\item SQL-Dump mit Größe von circa $13$ Gigabyte
		\item Einteilung in konvertierte und nicht-konvertierte Funnels
		\item Kampagnen in Form einer Baumstruktur organisiert
		\item Festlegung auf $17$ Kategorien
		\item \textit{Views} liegen in den nicht-konvertierten Funnels nur vor, wenn diese bei einem anderen Kunden der Refined Labs GmbH konvertiert sind
		\item $ 297,963 $ \textit{Clicks} für die konvertierten und $ 9,550,802 $ \textit{Clicks} für die nicht-konvertierten Funnels
		\item Erstellung von Features
	\end{itemize}
\end{frame}

\subsection{Views in den konvertierten Funnels}

\begin{frame}\frametitle{clickCount}
	\begin{columns}
		\column{7cm}
			\includegraphics[scale=0.39]{clickCountSucc.pdf}
		\column{4cm}
			\begin{itemize}
				\item Anzahl der Clicks bis zur aktuellen Position
				\item Gemittelt über alle konvertierten Funnels
				\item Mehr \textit{Views} als \textit{Clicks}
			\end{itemize}
	\end{columns}
\end{frame}

%\begin{frame}\frametitle{hasClicked}
	    %\centering\includegraphics[scale=0.39]{hasClickedSucc.pdf}
%\end{frame}

\begin{frame}\frametitle{Beschreibung der Kampagnen}
	\begin{table}[H]
		\tiny
		\begin{center}
			\begin{tabular}{|l|p{7cm}|}
				\hline \textbf{Kampagne} & \textbf{Beschreibung}\\ \hline
				\hline Affiliate - Partnerprogramm & Partner, die Werbemittel einbinden\\
				\hline Affiliate - Rest & Partner, die Zinsvergleich bereitstellen\\ 
				\hline Direct & Direkte Eingabe von \textit{www.interhyp.de}\\ 
				\hline Display & Bannerschaltungen\\
				\hline E-Mailing & Mails an Interessenten, die schon einen Antrag o.ä. gestellt haben\\
				\hline Generic & Unbezahlter Link\\
				\hline Kooperationen - Focus & \multirow{5}{7cm}{Individuelle Zusammenarbeit mit größeren Partnern}\\
				Kooperationen - Immonet & \\
				Kooperationen - Immoscout24 & \\
				Kooperationen - Immowelt & \\
				Kooperationen - Rest & \\
				\hline Newsletter & Regelmäßige Rundschreiben\\
				\hline SEM - Brand & \multirow{3}{7cm}{Bezahlte Suchergebnisse}\\
				SEM - Remarketing & \\
				SEM - Generisch & \\
				\hline SEO & Unbezahlte Suchergebnisse\\
				\hline Social Media & \textit{facebook} und \textit{gutefrage.net}\\
				\hline
			\end{tabular} 
		\end{center}
	\end{table}
\end{frame}

\begin{frame}\frametitle{campaign}
	\begin{columns}
		\column{7cm}
			\includegraphics[scale=0.39]{campaignSucc.pdf}
		\column{4cm}
			\begin{itemize}
				\item Hauptsächlich \textit{Display} bei Berücksichtigung der \textit{Views}
				\item Ausgewogenere Verteilung wenn \textit{Views} gelöscht werden
			\end{itemize}
	\end{columns}
\end{frame}

\subsection{Vergleich von konvertierten und nicht-konvertierten Funnels}

%\begin{frame}\frametitle{weekday}
	    %\centering\includegraphics[scale=0.3]{weekday.pdf}
%\end{frame}

%\begin{frame}\frametitle{hour}
	    %\centering\includegraphics[scale=0.3]{hour.pdf}
%\end{frame}

\begin{frame}\frametitle{campaign}
	\begin{columns}
		\column{7cm}
			\includegraphics[scale=0.39]{campaign.pdf}
		\column{4cm}
			\begin{itemize}
				\item \textit{Direct} am häufigsten in den konvertierten Funnels
				\item \textit{Display} und \textit{Affiliate - Partnerprogramm} am häufigsten in den nicht-konvertierten Funnels
			\end{itemize}
	\end{columns}
\end{frame}

\begin{frame}\frametitle{funnelLength}
	\begin{columns}
		\column{7cm}
			\includegraphics[scale=0.39]{funnelLength_First.pdf}
		\column{4cm}
			\begin{itemize}
				\item Anzahl Kontaktpunkte eines Funnels
				\item Kurze Funnels überwiegen deutlich
			\end{itemize}
	\end{columns}
\end{frame}

\begin{frame}\frametitle{timeSinceFirst}
	\begin{columns}
		\column{7cm}
			\includegraphics[scale=0.39]{timeSinceFirst_Last.pdf}
		\column{4cm}
			\begin{itemize}
				\item Verstrichene Zeit seit dem ersten Kontaktpunkt
				\item In der Abbildung wird nur der letzte Kontaktpunkt berücksichtigt
				\item Funnels mit Länge $1$ unberücksichtigt
				\item \textit{timeSinceLast}: Verstrichene Zeit seit dem vorherigen Kontaktpunkt
			\end{itemize}
	\end{columns}
\end{frame}

%\begin{frame}\frametitle{timeSinceLast}
	%\begin{columns}
		%\column{7cm}
			%\includegraphics[scale=0.39]{timeSinceLast.pdf}
		%\column{4cm}
			%\begin{itemize}
				%\item 
			%\end{itemize}
	%\end{columns}
%\end{frame}

\begin{frame}\frametitle{freq}
	\begin{columns}
		\column{7cm}
			\includegraphics[scale=0.39]{freq.pdf}
		\column{4cm}
			\begin{itemize}
				\item \textit{funnelLength} dividiert durch Gesamt-Beobachtungsdauer in Stunden
				\item Frequenzen in nicht-konvertierten Funnels höher
			\end{itemize}
	\end{columns}
\end{frame}

\section{Methoden}

\begin{frame}\frametitle{Inhalt}
	\tableofcontents[currentsection,hideallsubsections]
\end{frame}

\subsection{Zeitdiskretes Survival-Modell}

\begin{frame}
	\begin{itemize}
		\item Zeit bis zu einem Ereignis $\Rightarrow$ Konvertierung oder Nicht-Konvertierung bzw. Rechtszensierung
		\item Positionen bilden Zeitachse des Modells $\Rightarrow$ Zeitdiskretes Modell
		%\item Stochastic Gradient Boosting mit Stümpfen als Basis-Lerner
		\item Zielvariable:	
			\begin{align*}
				y_{ip} =& \begin{cases} 1 & \text{Beobachtung } i \text{ konvertiert an Position } p\\
															 0 & \text{sonst} 
								 \end{cases}\\
								&p=1,...,25 \text{, } i=1,...,N_p 
			\end{align*}	
	\end{itemize}
\end{frame}

\begin{frame}
	\begin{itemize}
		%\item Zielvariable:	
			%\begin{align*}
				%y_{ip} =& \begin{cases} 1 & \text{Beobachtung } i \text{ konvertiert an Position } p\\
					%										 0 & \text{sonst} 
						%		 \end{cases}\\
							%	&p=1,...,25 \text{, } i=1,...,N_p 
			%\end{align*}	
		\item Hazardrate: 
			\begin{align*}
				\lambda_{ip} = P(y_{ip}=1|funnelLength_i \geq p, x_{ip})
			\end{align*}
		\item Logit-Modell: 
			\begin{align*}
				y_{ip}|x_{ip} &\stackrel{ind}{\sim} Bin(1, \lambda_{ip})  \\
				E(y_{ip}|x_{ip}) = P(y_{ip} = 1|x_{ip}) = \lambda_{ip} &= h(f_p(x_{ip})) = \frac{\exp(f_p(x_{ip}))}{1+\exp(f_p(x_{ip}))}
			\end{align*}
		\item Binomieller Verlust: 
			\begin{align*}
				L(y_{ip},f_p(x_{ip})) = -\sum_{i=1}^{N_p} (y_{ip} f_p(x_{ip}) + \ln(1+\exp(f_p(x_{ip}))))
			\end{align*}
	\end{itemize}
\end{frame}

%\begin{frame}
	%\begin{itemize}
		%\item Likelihood: 
			%\begin{align*}
				%L(\lambda_{ip}) &= \prod_{i=1}^{N_p} \lambda_{ip}^{y_{ip}} (1-\lambda_{ip})^{1-y_{ip}}
			%\end{align*}
		%\item Log-Likelihood: 
			%\begin{align*}
				%l(\lambda_{ip}) &= \ln(L(\lambda_{ip})) = \sum_{i=1}^{N_p} (y_{ip} \ln(\lambda_{ip}) + (1-y_{ip}) \ln(1-\lambda_{ip}))\\
				%&= \sum_{i=1}^{N_p} (y_{ip} f(x_{ip}) - \ln(1+\exp(f(x_{ip}))))
			%\end{align*}
		%\item Binomieller Verlust: 
			%\begin{align*}
				%L(y,f) = -yf + \ln(1+\exp(f))
			%\end{align*}
	%\end{itemize}
%\end{frame}

\begin{frame}
	\begin{itemize}
		\item Prädiktorfunktion:
			\begin{align*}
			f_p(x_{ip}) =&f_{weekday,p}(\text{weekday}_{ip}) +\\
								 &f_{hour,p}(\text{hour}_{ip}) +\\
								 &f_{campaign,p}(\text{campaign}_{ip}) +\\
								 &f_{campaignLast,p}(\text{campaign}_{i,p-1}) +\\
								 &f_{campaignLast2,p}(\text{campaign}_{i,p-2}) +\\
								 &f_{timeSinceLast,p}(\text{timeSinceLast}_{ip}) +\\
								 &f_{timeSinceFirst,p}(\text{timeSinceFirst}_{ip}) +\\
								 &\text{offset}(\hat{\lambda}_{i,p-1})
			\end{align*}
	\end{itemize}
\end{frame}

\begin{frame}\frametitle{Gradient Boosting - Pseudocode}
	\floatname{algorithm}{Algorithmus}
	%\begin{algorithm}
	%\caption{Gradient Boosting}\label{alg}
	%\label{gradboosting}
		\begin{algorithmic}
		\STATE Setze Startwert für $f_{0p}(x_{ip})$
		\FOR{$m=1:n.trees$}
			\STATE Setze $\lambda_{ip}(x_{ip}) = \frac{\exp(f_{m-1,p}(x_{ip}))}{1+\exp(f_{m-1,p}(x_{ip}))}$
			\FOR{$i=1:N_p$} 
				\STATE $r_{imp} = - \frac{\partial L(y_{ip},f_{m-1,p}(x_{ip}))}{\partial f_{m-1,p}(x_{ip})} = y_{ip} - \lambda_{ip}(x_{ip})$
			\ENDFOR
			%\STATE Fit a regression base learner to the pseudo-residuals $r_{im}$:
			\STATE $\theta_{mp} = \argmin_{\theta} \sum_{i=1}^{N_p} (r_{imp} - h(x_{ip}, \theta))^2$
			\STATE $\beta_{mp} = \argmin_{\beta} \sum_{i=1}^{N_p} L(y_{ip}, f_{m-1,p}(x_{ip}) + \beta h(x_{ip},\theta_{mp}))$
			\STATE $f_{mp}(x_{ip}) = f_{m-1,p}(x_{ip}) + \beta_{mp} h(x_{ip},\theta_{mp})$
		\ENDFOR
		\end{algorithmic}
	%\end{algorithm}
\end{frame}

\begin{frame}\frametitle{Parameter des Modells}
	\begin{itemize}
		\item Trainingsdaten machen Hälfte der gesamten Daten aus - stratifiziert bezüglich Transaction, Campaign, funnelLength
		\item $n.trees=3000$
		\item $cv.folds=5$
		\item Shrinkage-Parameter: $\mu = 0.01 \Rightarrow f_{mp}(x_{ip}) = f_{m-1,p}(x_{ip}) + \mu \beta_{mp} h(x_{ip},\theta_{mp})$
		\item $interaction.depth=1$
		\item $bag.fraction=0.5 \Rightarrow$ \textbf{Stochastic} Gradient Boosting
	\end{itemize}
\end{frame}

\begin{frame}\frametitle{Output des Modells}
	\begin{itemize}
		\item $\hat{f}_p(x_{ip})$ für jede Beobachtung $i$ und jede Position $p$
			\begin{align*}
				\hat{\lambda}_{ip} = \frac{\exp(\hat{f}(x_{ip}))}{1+\exp(\hat{f}(x_{ip}))}
			\end{align*}
		\item Relative Wichtigkeit der Features:
			\begin{align*}
				\hat{I}_{jp} &= \sqrt{\frac{1}{M} \sum_{m=1}^{n.trees} \hat{i}_{mp} 1_{jmp}}
			\end{align*}
		\item Marginale Effekte der Features:
			\begin{align*}
				\bar{f}_{jp}(x_{jp}) = \frac{1}{N} \sum_{i=1}^{N_p} \hat{f}(x_{jp},x_{i,\backslash j,p})
			\end{align*}
		\item ROC-Kurve und AUC
	\end{itemize}
\end{frame}

\subsection{Sequential Pattern Mining}

\begin{frame}
	\begin{itemize}
		\item Menge von Items $I = \{a, b, c, d, e\}$ $\Rightarrow$ Kampagnen
		\item Datenbank: $[ID 1, <abcdbaae>]$; $[ID 2, <edcaa>]$
		\item 4-Sequenz $s = b\rightarrow b\rightarrow a\rightarrow e$
		\item Support einer Sequenz: Anteil der IDs, die $s$ unterstützen
		\item SPADE-Algorithmus findet häufige Sequenzen, deren Support größer als ein festgelegter minimaler Support ist
		\item Seperate Anwendung auf konvertierte und nicht-konvertierte Funnels
	\end{itemize}
\end{frame}

\subsection{Visualisierung anhand eines Netzwerkes}

\begin{frame}
	\begin{itemize}
		\item Geordneter Graph $G=(V,E)$ besteht aus Menge $V$ von Knoten und Menge $E$ von Kanten
		\item Kante $e_i \in E$ besteht aus geordneten Paar von zwei Knoten $(v_j,v_k)$, wobei $v_j,v_k \in V$
		\item Startpunkt $\rightarrow$ $17$ Kampagnen der ersten Position $\rightarrow$ $Succ\_1$, $Fail\_1$ und $17$ Kampagnen der zweiten Position $\rightarrow$ $Succ\_2$, $Fail\_2$ und $17$ Kampagnen der dritten Position $\rightarrow$ ...
		\item Kanten sind bezüglich der Anzahl der Nutzer gewichtet
		\item Relative Ausgänge: relative Häufigkeiten der Kanten, wobei die zugrundeliegende Menge die Summe aller Nutzer ist, die einen Knoten verlassen
		\item Relative Eingänge: relative Häufigkeiten der Kanten, wobei die zugrundeliegende Menge die Summe aller Nutzer ist, die in einen Knoten gehen
	\end{itemize}
\end{frame}

\begin{frame}
	\begin{itemize}
		\item \textit{R}-Paket \textit{rgexf} $\rightarrow$ \textit{gexf}-Datei $\rightarrow$ \textit{Gephi}
		\item Berechnung der räumlichen Anordnung der Knoten und Kanten anhand von Algorithmen (z.B. \textit{Force Atlas 2})
		\item Manuelle Bearbeitung für die Präsentation von Ergebnissen
		\item Interaktives Arbeiten mit dem Netzwerk in \textit{Gephi} möglich $\rightarrow$ Tutorial dazu im Bericht
	\end{itemize}
	\centering\includegraphics[scale=0.2]{graphbegin.pdf}
\end{frame}

\section{Ergebnisse}

\begin{frame}\frametitle{Inhalt}
	\tableofcontents[currentsection,hideallsubsections]
\end{frame}

\subsection{Zeitdiskretes Survival-Modell}

%\begin{frame}\frametitle{Optimale Iterationsanzahl}
	%\centering\includegraphics[scale=0.3]{bestIter.pdf}
%\end{frame}

\begin{frame}\frametitle{Relative Wichtigkeit der Features}
	\centering\includegraphics[scale=0.39]{variableImportance.pdf}
\end{frame}

\begin{frame}\frametitle{Marginale Effekte - campaign}
	\centering\includegraphics[scale=0.39]{marg_eff_campaign.pdf}
\end{frame}

%\begin{frame}\frametitle{Marginale Effekte - campaignLast}
	%\centering\includegraphics[scale=0.39]{marg_eff_campaignLast.pdf}
%\end{frame}

\begin{frame}\frametitle{Marginale Effekte - timeSinceFirst \& timeSinceLast}
	\centering\includegraphics[scale=0.39]{marg_eff_time.pdf}
\end{frame}

%\begin{frame}\frametitle{ROC-Kurve}
	%\centering\includegraphics[scale=0.39]{roc.png}
%\end{frame}

\begin{frame}\frametitle{AUC}
	\centering\includegraphics[scale=0.39]{auc.pdf}
\end{frame}

\subsection{Sequential Pattern Mining}

%\begin{frame}\frametitle{Häufige Sequenzen in konvertierten und nicht-konvertierten Funnels}
	%\centering\includegraphics[scale=0.3]{spm_all.pdf}
%\end{frame}

\begin{frame}
	\begin{columns}
		\column{7cm}
			\includegraphics[scale=0.3]{spm_min15.pdf}
		\column{4cm}
			\begin{itemize}
				\item Kurze Funnels überwiegen
				\item Nur Funnels mit mindestens $15$ Kontaktpunkten berücksichtigt
				\item Minimaler Support: $0.2$
			\end{itemize}
	\end{columns}
\end{frame}

\subsection{Visualisierung anhand eines Netzwerkes}

\begin{frame}\frametitle{Netzwerk für die ersten $10$ Positionen}
	\centering\includegraphics[scale=.55]{netzwerk_length_10.pdf}
\end{frame}


\begin{frame}\frametitle{Relative Ausgänge}
	\centering\includegraphics[scale=1.75]{out_labels.pdf}
\end{frame}

\begin{frame}\frametitle{Relative Ausgänge mit Filter $0.02$}
	\centering\includegraphics[scale=0.3]{out_filter_2_succ.png}
\end{frame}

\begin{frame}\frametitle{Relative Ausgänge mit Filter $0.5$}
	\centering\includegraphics[scale=0.3]{out_filter_50_fail.png}
\end{frame}

\begin{frame}\frametitle{Relative Eingänge}
	\centering\includegraphics[scale=0.75]{in_labels.pdf}
\end{frame}

\begin{frame}\frametitle{Relative Eingänge mit Filter $0.1$}
	\centering\includegraphics[scale=0.3]{in_filter_10_succ.png}
\end{frame}

\begin{frame}\frametitle{Relative Eingänge mit Filter $0.1$}
	\centering\includegraphics[scale=0.3]{in_filter_10_fail.png}
\end{frame}

\section{Zusammenfassung}

\begin{frame}\frametitle{Inhalt}
	\tableofcontents[currentsection,hideallsubsections]
\end{frame}

\begin{frame}\frametitle{Zusammenfassung der Ergebnisse}
	\begin{itemize}
		\item Zeitdiskretes Survival-Modell
		\begin{itemize}
			\item Klassifikation in konvertierte und nicht-konvertierte Funnels
			\item Marginale Effekte der Features
		\end{itemize}
		\item Sequential Pattern Mining
		\item Netzwerk
		\begin{itemize}
			\item Visualisierung der gesamten Daten
			\item Bestätigung der Ergebnisse aus Survival-Modell und Sequential Pattern Mining
			\item Tutorial zum interaktiven Arbeiten im Bericht
		\end{itemize}
	\end{itemize}
\end{frame}

\begin{frame}{Literatur}
	\begin{thebibliography}{1}
		\bibitem{} J. H. Friedman \& B. E. Popescu (2005): {\glqq Predictive Learning via Rule Ensembles\grqq}.
		\bibitem{} R. Agrawal \& R. Srikant (1995): {\glqq Mining sequential patterns\grqq}.
		\bibitem{} M. J. Zaki (2001): {\glqq SPADE: An efficient algorithm for mining frequent sequences\grqq}.
		\bibitem{} M. Bastian, S. Heymann \& M. Jacomy. (2009): {\glqq Gephi: An Open Source Software for Exploring and Manipulating Networks\grqq}.
		\bibitem{} R-Packages: \textit{gbm, rgexf, arulesSequences, data.table, plyr, ggplot2, doSNOW, foreach}.
	\end{thebibliography}
\end{frame} 

\begin{frame}
	\centering \huge
	Vielen Dank für Ihre Aufmerksamkeit!
\end{frame}

\end{document}

