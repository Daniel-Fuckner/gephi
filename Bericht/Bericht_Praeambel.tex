\documentclass[12pt,a4paper]{scrartcl} 

%Deutsch:
	\usepackage[ngerman]{babel} %Deutsches Datumformat, Umlaute m\"oglich,...
	\usepackage[utf8]{inputenc}
	\usepackage[T1]{fontenc} 
	
%Quellenverzeichnis:
	\usepackage[maxcitenames=2,autocite=footnote,uniquename=full,uniquelist=true,backend=biber]{biblatex} %style=authoryear-icomp
	\usepackage{csquotes} %Hilfspaket für Biblatex
	\bibliography{bib_database.bib} %Datei mit bibliographischen Daten
	\DefineBibliographyStrings{ngerman}{andothers={et\ al\adddot}} % "u.a." zu "et al."
	\DefineBibliographyStrings{ngerman}{and={\&}} % "und" zu "&"

%Mathematik:
	\usepackage{dsfont} %Symbole
	\usepackage{amsmath} %Umgebung
	\usepackage{amssymb} %Symbole
	\usepackage{bbm} %doppelstreifen bei buchstaben (zb symbol für ganze zahlen \mathbbm{Z})
	
%Grafiken:
	\usepackage{graphics}
	\usepackage{graphicx}
	\usepackage{picinpar} %bilder so einfügen, dass text um bilder weiterläuft
	\usepackage{float} %\begin{figure}[H] => Grafik wird HIER eingefügt!

%Euro-Zeichen:
	\usepackage{eurosym}

%Pseudo Code:
	\usepackage{algorithmic}
	\usepackage{algorithm}
	\renewcommand{\algorithmicrequire}{\textbf{Eingabe:}}
	\renewcommand{\algorithmicensure}{\textbf{Ausgabe:}}

%Verzeichnis-struktur darstellen:
	\usepackage{dirtree}

%Tabellen:
	\usepackage{multirow}
	
%Quellcode:
	%\usepackage[numbered,framed]{mcode} %Quellcode darstellen
	
%Englisch:
	%\usepackage[ngerman,english]{babel} %automatisch erzeugte Überschriften etc. auf englisch
	
\DeclareMathOperator*{\argmin}{arg\,min}